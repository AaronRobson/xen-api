\documentclass[a4paper,12pt]{article}

\usepackage[utf8]{inputenc}
\usepackage{fullpage,url}
\usepackage{algorithm}
\usepackage{algorithmic}
\usepackage{listings}
\usepackage{amssymb,amsmath,amsthm}
\usepackage{color}

\title{A Formal Proof of Bellman-Ford Algorithm}
\author{Yuto Takei \\ The University of Tokyo }

\lstdefinelanguage{Why3}
{
basicstyle=\ttfamily,
keywordstyle=\color{blue},
morekeywords=[1]{inductive,predicate,function,goal,type,use,import,theory,end,in,axiom,lemma,export,forall,constant,module,let,exception,match,with,exists},%
%keywordstyle=[1]{\color{red}},%
morekeywords=[2]{true,false},%
%keywordstyle=[2]{\color{blue}},%
otherkeywords={},%
commentstyle=\itshape,%
columns=[l]fullflexible,%
sensitive=true,%
morecomment=[s]{(*}{*)},%
escapeinside={*?}{?*},%
keepspaces=true,
literate=%
{'a}{$\alpha$}{1}%
{'b}{$\beta$}{1}%
{<}{$<$}{1}%
{>}{$>$}{1}%
{<=}{$\le$}{1}%
{>=}{$\ge$}{1}%
{<>}{$\ne$}{1}%
{/\\}{$\land$}{1}%
{\\/}{ $\lor$ }{3}%
{\ or(}{ $\lor$(}{3}%
{not\ }{$\lnot$ }{1}%
{not(}{$\lnot$(}{1}%
{->}{$\rightarrow$}{2}%
{<->}{$\leftrightarrow$}{2}%
}

\lstnewenvironment{why3}
{\lstset{language=Why3}}
{}

\newcommand{\edgen}[3]{#1\rightarrow_{#3}#2}
\newcommand{\pathn}[3]{#1\rightarrow^\star_{#3}#2}

\begin{document}

\maketitle

\begin{abstract}
  This is the abstract.
\end{abstract}

\section{Introduction}

I want to \emph{emphasize} a word.

\section{Bellman-Ford Algorithm}

The Bellman-Ford algorithm is one of the algorithms to solve the
single-source shortest paths problem, i.e., to give the shortest paths
to all vertices from one given source vertex. To solve such a problem,
Dijkstra's algorithm is widely known as an effective solution, yet it
assums edges to have non-negative length. The Bellman-Ford algorithm
relaxed such the assumption and is able to compute the solution for
directed graphs with edges that have negative length.

However on the other hand, if the graph contains a cycle whose total
length of involving edges is negative, it may cause an inability to
compute the shortest paths. This is only limited to the case such a
negative-cylce is reachable from the given source, and the
Bellman-Ford algorithm detects and reports it without failure.

Now we give the formal description of the single-source shortest
paths. It is given as a triple $ (G,s,l) $, where $ G $ represents a
directed graph $ G=(V,E) $, $ s \in V $ is a source vertex, and $ l :
E \rightarrow \mathbf{R} $ gives the length for all edges. The goal
for this problem is either to produce the shortest path tree from $ s
$ or to prove that there is a negative-cycle reachable from $ s $.

The listing \ref{alg:bf} shows the psuedocode of the Bellman-Ford
algorithm \cite[p.~532]{algo}.

\newcommand{\initBF}{\ensuremath{\mbox{\sc Initialize-Single-Source}}}
\newcommand{\relaxBF}{\ensuremath{\mbox{\sc Relax}}}
\newcommand{\mainBF}{\ensuremath{\mbox{\sc Bellman-Ford}}}

\begin{algorithm}
\caption{$ \initBF (G,s) $}
\begin{algorithmic}[1]
\FORALL{ vertex $ v \in V[G] $}
\STATE{$ d[v] \leftarrow \infty $}
\STATE{$ \pi [v] \leftarrow nil $}
\ENDFOR
\STATE{$ d[s] \leftarrow 0 $}
\end{algorithmic}
\end{algorithm}

\begin{algorithm}
\caption{$ \relaxBF (u,v,l) $}
\begin{algorithmic}[1]
\IF{$ d[v]>d[u]+l(u,v) $}
\STATE{$ d[v] \leftarrow d[u]+l(u,v) $}
\STATE{$ \pi [v] \leftarrow u $}
\ENDIF
\end{algorithmic}
\end{algorithm}

\begin{algorithm}
\caption{$ \mainBF (G,s,l) $}\label{alg:bf}
\begin{algorithmic}[1]
\STATE{$ \initBF (G,s) $}
\FOR{$i = 1 \to |V[G]|-1 $}
\FORALL{ edge $ (u,v) \in E[G] $}
\STATE{$ \relaxBF (u,v,l) $}
\ENDFOR
\ENDFOR
\FORALL{edge $ (u,v)\in E[G] $}
\IF{$ d[v]>d[u]+l(u,v) $}
\RETURN{False}
\ENDIF
\ENDFOR
\RETURN{True}
\end{algorithmic}
\end{algorithm}


\section{Quick Overview of Why3}

Why3 is a software verification platfrom that enables one to describe
the algorithm properties by first-order logic. Its important feature
is the ability to separate the logical statements from programs
\cite{boogie11why3}.

\subsection{Logic}

Logic component is described by set of theories that contains the
definition of data types, their characteristics as either an axiom or
a lemma, and predicates (or functions if returns non-Boolean) upon the
given arguments.

\paragraph {[TODO: align the following paragraphs]} 

\emph{Types.} Types are either a non-interpreted abstract type or an
algebraic type with polymorphism supported.

\emph{Predicates and functions.} Predicate takes any number of
arguments with the above type and returns a Boolean value. Function
plays the same role but returns non-Boolean values. They are strictly
typed before verification and recognized its recursiveness. The other
important things is that the termination of them is guaranteed by the
automated check of the structural decreasing of arguments.

\emph{Inductive predicates.} Inductive predicate is provided as an
other option to declare predicates. It is described by the set of
conditional clauses where the predicate holds, and defined as the
minimum predicate that all of them hold.

\subsection{Program}

Program component takes over the similar syntax as Objective Caml
\ref{ocaml}, but we define conditions of the program additionally.

\emph{Pre-condition and post-condition.} Pre-condition is the
assumption that the algorithm makes prior to its execution, and the
post-condition is the result the aglrotihm ensures after its
execution. It is often useful to describe the effect of the
algorithm. One should give the most strict statement for
pre-condition, and the most general statement for post-condition.

\emph{Loop variant.} Variant expression is used to guarantees the
terminability of recursive calls or loops. One must give an arbitral
integral expression for it, which necessarily decreases in the certain
sequential operations. With the assumption that the given expression
remains positive during the iteration, and firmly decreases at every
step, Why3 guarantees that the iteration will eventually terminate.

(TODO: Insertion of small example)


\section{Formal Proof}

\subsection{Specification}

Proving the algorithm first starts by providing the specification of
the algorithm. It should be minimal yet comprehensive to define the
behavior of algorithm and should be clearly distinguished from the
proofs. We now present minimal logical properties of graph data
structure as prerequisites for Bellman-For algorithm, and then
introduce those of the algorithm itself.
 
\subsubsection{Graph}

In order to show the correctness of algorithms that make use of graph
data structure, it is neccesary to define the characteristics of graph
itself in purely logical way. There are several ways to represent the
graph for formal verification.

\begin{enumerate}

\item Defining the logical functions, which give the set of vertices,
  successors to each of them, and weight between one vertex and
  another. Note that the verification context encapsulate the actual
  graph and thus the algorithm should obtain it by calling those
  global functions.

\item Passing the graph as one of the input parameters for the
  algorithm. Although this structure is more natural in practical
  implementation, one must pass the graph to all involving
  sub-functions as a parameter and describe the desired property of it
  as a pre-condition for them.

\end{enumerate}

Note that both of above will eventually result in the similar
proof. For the simplicity in this paper, we choose the first method.

Considering the Bellman-Ford algorithm, one must give the formal
definition of the path with the given length. We define such the path
by inductive manner. Let $\edgen{v_1}{v_2}{n}$ denote the existence of
the edge between $ v_1 $ and $ v_2 $ with the length $ n $, and $ \to
^* $ represent the transitive closure of $ \to $; then following two
properties are introduced.

\begin{displaymath}
  \frac{}
       {\pathn{v}{v}{0}}
  \qquad\qquad
  \frac{\pathn{v_1}{v_2}{m} \qquad \edgen{v_2}{v_3}{n}}
       {\pathn{v_1}{v_3}{m+n}}
\end{displaymath}

(\textbf{TODO:} My arrow is so ugly... $ \stackrel{m+n}{\longrightarrow^*} $)

After introducing the predicate for the path existence between certain
vertices, one can immediately precise that the property of the
shortest path from $ v_1 $ to $ v_2 $ with length $ n $, i.e., to
assert that there are no path shorter than $ n $ connecting $ v_1 $ to
$ v_2 $. Moreover, if no such path exists from $ v_1 $ to $ v_2 $, it
is suffice to say that there is no $ n $, which is the length between
those vertices.

With the specification in Why3 syntax, the formulation above is
expressed as in the listing~\ref{lst:why_graph}. Here all graph
structures and predicates are packed in the theory \mbox{\sc
  Graph}. Vertices are given as a global constant set namely \mbox{\sc
  vertices}, and edges and their weights are given by global functions
\mbox{\sc succ} and \mbox{\sc weight} respectively.

\begin{algorithm}
\caption{Definition of graph in logical way}\label{lst:why_graph}
\begin{why3}[1]
theory Graph
  use export int.Int
  use export set.Fset

  type vertex
  constant vertices : set vertex
  function succ vertex : set vertex
  function weight vertex vertex : int

  inductive path (v1 v2: vertex) (n: int) =
    | path_empty:
        forall v: vertex. path v v 0
    | path_succ:
        forall v1 v2 v3: vertex, n: int.
        path v1 v2 n -> mem v3 (succ v2) -> path v1 v3 (n + weight v2 v3)

  predicate shortest_path (v1 v2: vertex) (n: int) =
    path v1 v2 n /\
    forall m: int. m < n -> not (path v1 v2 m)

  predicate no_path (v1 v2: vertex) =
    forall n: int. not (path v1 v2 n)
end
\end{why3}
\end{algorithm}

\subsubsection{Bellman-Ford algorithm}

Now we discuss the foundamental characteristics of Bellman-Ford
algorithm. They consists of precondition, postcondition and
terminability of the algorithm, and we aim to describe it by utilizing
Hoare logic.

Although the psuedocode of Bellman-Ford algorithm in listing
\ref{alg:bf} returns an Boolean value and modify the distance and
parent mappings in imperatiev manner, we slightly change the format
for practical implementation in O'Caml and let it return a mapping
from vertex to shortest-distances rather than the indication of the
existence of negative-cycle. Additionally, we raise an exception in
presence of the negative-cycle.

As a result, our implementation will be a function that takes the
source vertex $ s $ as a single argument, and either returns the
mapping from $ v $ to \mbox{\sc int} for shortest paths, or raise an
exception indicating the presence of a negative cycle. Now we adapt
algorithmic characteristic into the implementation above.

\emph{Precondition.} The only requirement is that the source vertex $
s $ given must be contained in the graph $ G $, which is described by
the theory \mbox{\sc Graph}.

\emph{Postcondition.} If algorithm returns the value without the
exception, the resulting mapping at the key $ v $ should give the
length of the shortest path from $ s $ to $ v $. The existence of such
a path is indicated by the algebraic type, which either has a finite
integer value or an inifinity. With the exception returned, there
exists at least one negative-cycle.

\emph{Terminability.} (TODO)

Note that the case without the exception, one does not need to assert
the graph to be free of negative-cycle.








\begin{algorithm}
\caption{Specification of Bellman-Ford algorithm}\label{lst:why_bf}
\begin{why3}[1]
module BellmanFord
  use import map.Map
  use import Graph

  type dist = Finite int | Infinite
  type distmap = map vertex dist

  exception NegativeCycle

  let bellman_ford (s: vertex) : distmap =
    { mem s vertices }
    (* ... actual implementation comes here ... *)
    { forall v: vertex. mem v vertices ->
        match result[v] with
        | Finite n -> shortest_path s v n
        | Infinite -> no_path s v
        end }
    | NegativeCycle ->
    { exists v: vertex. mem v vertices /\
      exists n: int. n < 0 /\ path v v n  }
end
\end{why3}
\end{algorithm}




\subsection{Proof}

\section{Conclusion}

\bibliographystyle{plain}
\bibliography{./biblio}

\end{document}
